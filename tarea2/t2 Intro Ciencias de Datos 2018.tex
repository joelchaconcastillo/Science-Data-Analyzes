\documentclass{article}
\usepackage{amsfonts}
\usepackage{amsmath}
\usepackage{amssymb}
\usepackage{pictex}
\usepackage[dvips]{graphics}
\usepackage{color}
\usepackage{psfrag}
\usepackage{verbatim}
\usepackage{enumerate}

\newcommand {\bc}{\begin{center}}
\newcommand {\ec}{\end{center}}


\pagestyle{empty}
\setlength{\oddsidemargin}{0pt}
\setlength{\textwidth}{463pt}
\setlength{\marginparsep}{0pt}
\setlength{\marginparwidth}{60pt}
\setlength{\topmargin}{0pt}
\setlength{\headheight}{0pt}
\setlength{\headsep}{0pt}
\setlength{\textheight}{650pt}
\setlength{\footskip}{0pt}

\begin{document}
\begin{sf}
\bc
{\Large Tarea 2. Introducci\'on a Ciencias de Datos 2018}
\ec

\bigskip

\begin{enumerate}
% 1 Ripley, p.22
\item Considere tres poblaciones Poisson, con par\'ametros
$\lambda_1=10$, $\lambda_2=15$ y $\lambda_3=20$ respectivamente.
\begin{enumerate}
\item Establezca la regla \'optima de clasificaci\'on,
bas\'andose en una s\'ola observaci\'on, $x$.
\item Calcule la probabilidad de error asociado a esta regla \'optima.
\item Escriba un programa (R o Python) para validar, v\'ia simulaci\'on,
el nivel de error encontrado en el inciso anterior.
\item Suponga ahora que, en vez de hacer una s\'ola observaci\'on
sobre el objeto a clasificar, se hacen dos observaciones independientes,
$x_1$ y $x_2$. Encuentre la regla \'optima de clasificaci\'on
pero ahora basada en $\bar{x} = (x_1+x_2)/2$.
\item ?`Qu\'e tanto mejora el procedimiento, comparado con
el basado en una sola observaci\'on?.
\end{enumerate}


\bigskip

%2
\item Ingrese a la p\'agina del repositorio de datos de Machine
Learning de la Univ. de 
California en Irvine:
\begin{center}
\texttt{https://archive.ics.uci.edu/ml/index.php}
\end{center}
\begin{itemize}
\item Bajar el conjunto de datos "Wine"
\item Basados en las caracter\'isticas qu\'imicas de los vinos,
construya un clasificador para determinar el origen de los mismos.
\end{itemize}




\bigskip




\item Suponga que $x$ es un vector aleatorio $d$-dimensional, con
media $\mu$ y varianza $\Sigma$ y se tienen las siguientes particiones:
$$x = \left[\begin{array}{c} x_1 \\ x_2\end{array}\right], \qquad
\mu = \left[\begin{array}{c} \mu_1 \\ \mu_2\end{array}\right], \qquad
\Sigma = \left[\begin{array}{cc} \Sigma_{11} & \Sigma_{12} \\ \Sigma_{21} & \Sigma_{22}\end{array}\right]$$ 
donde $x_1$ es $d_1 \times 1$ y $x_2$ es $d_2 \times 1$, con $d_1+d_2=d$
(suponga adem\'as que $\Sigma_{11}$ y $\Sigma_{22}$ son positivas definidas). 
Considere $y_1=a_1^T x_1$ y $y_2=a_2^T x_2$, donde $a_1$ y $a_2$ son
vectores no aleatorios de dimensiones $d_1 \times 1$ y $d_2 \times 1$ respectivamente.
?`Qu\'e valores de $a_1$ y $a_2$ maximizan el cuadrado de la correlaci\'on
entre $y_1$ y $y_2$?
\begin{itemize}
\item Nota 1: Este problema es llamado ```problema de correlaci\'on can\'onica''.
\item Nota 2: Recuerde que la correlaci\'on entre dos variables aleatorias univariadas se define como
$$\text{Corr}(u,v) = \frac{\text{Cov}(u,v)}{\sqrt{\text{Var}(u)}  \; \sqrt{\text{Var}(v)}}$$
\item Nota 3: Si $B$ es positiva definida y $a$ es un vector, entonces
$$\underset{x}{\text{sup}} \; \frac{(a^Tx)^2}{x^TBx} = a^TB^{-1}a$$
y el supremo se alcanza cuando $x$ es proporcional a $B^{-1}a$.
Este resultado es un caso especial del resultado visto en clase acerca de
la maximizaci\'on del cociente de Rayleigh (cociente de dos formas cuadr\'aticas).
\item Nota 4: Sugerencia
$$\underset{a_1,a_2}{\text{sup}} ( \cdot ) = \underset{a_1}{\text{sup}} 
\left\{ \underset{a_2}{\text{sup}} (\cdot ) \right\}$$
\end{itemize}



\bigskip





\item Sean $y_1(x), \ldots, y_K(x)$, $K$ funciones lineales de $x$,
esto es $y_j(x) = w_j^Tx+w_{j0}$, donde las $w_j$'s son vectores y las
$w_{j0}$'s son escalares. Sean
$$R_j = \left\{\; x \; | \; y_j(x) \ge y_r(x), \text{ para todo } r \ne j \; \right\}, \quad j=1,\ldots , K$$
Muestre que $R_j$, $j=1,\ldots , K$ son conjuntos convexos.
En el contexto de clasificadores lineales, esto quiere decir que
las regiones de asignaci\'on para cada grupo, son regiones convexas.
Los procedimientos de clasificaci\'on no siempre inducen regiones convexas
(e.g. $k$-nn, como veremos). ?`Es deseable tener regiones convexas? (explicar).



\bigskip


\item Considere un conjunto de datos $x_1,\ldots,x_n$,
los cu\'ales pertenecen a $K$ grupos o poblaciones $C_1,\ldots,C_K$, con $n=n_1+\ldots+n_K$, donde
$n_i = \# \{C_i\}$. Definamos
$$\bar{x}_i = \frac{1}{n_i}\sum_{x \in C_i} x, 
\qquad \bar{\bar{x}} = \frac{1}{n} \sum_{j=1}^n x_j, 
\qquad S_i = \sum_{x \in C_i} (x-\bar{x}_i)(x-\bar{x}_i)^T, 
\qquad S_W = \sum_{i=1}^K S_i$$
$$S_B = \sum_{i=1}^K n_i (\bar{x}_i-\bar{\bar{x}})(\bar{x}_i-\bar{\bar{x}})^T,
\qquad S_T = \sum_{x \in D} (x-\bar{\bar{x}})(x-\bar{\bar{x}})^T$$
Puede verse que la matriz de dispersi\'on total, $S_T$,  es la suma de la matriz 
de dispersi\'on dentro de grupos, $S_W$, y la matriz de dispersi\'on entre grupos, $S_B$,
esto es, $S_T = S_W+S_B.$
Considere las siguientes cantidades: 
$$L_1 = \text{tr}\left( S_T^{-1}S_W \right), \quad \qquad L_2 = \frac{|S_W|}{|S_T|}.$$
Estas han sido usadas como criterios para determinar
que tan separados est\'an los grupos. Sean $\lambda_1,\ldots,\lambda_d$ los valores propios de $S_W^{-1}S_B$.
\textbf{muestre que}:
$$L_1 = \sum_{i=1}^d \frac{1}{1+\lambda_i}, \qquad L_2 = \prod_{i=1}^d \frac{1}{1+\lambda_i}.$$
Ayuda: Mostrar que $S_T^{-1}S_W = (I+S_W^{-1}S_B)^{-1}$ y esto ayuda a resolver el problema. 

\end{enumerate}

\bigskip

\underline{\hspace{15cm}}

\bigskip

\noindent \textbf{Fecha de entrega: Jueves 30 de agosto (entregar s\'olo los problemas 1, 2 y 3).}




\end{sf}
\end{document}
